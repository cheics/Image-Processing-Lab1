\documentclass[article, 1.5space, letterpaper, 12pt, oneside, header, footer]{SydeClass}
\graphicspath{{images/}}
\usepackage{subfigure}
\usepackage{eqnarray}


% --------- Title Info -----------
\titlestyle{design} % used in SydeTitle.tex. Can equal one of the following values: design, work

\title{Lab 1}
\subtitle{Fundamentals of Image Processing}

\coursecode{SYDE 475}
\department{Systems Design Engineering}

\author{Colin Heics, 20240543}
\authorheader{C. Heics}
\authortwo{Neil Sokol, 20265064}
\authorheadertwo{N. Sokol}

\date{\today}
\instructor{Alex Wong}

\subsectionfont{\normalsize}
\setcounter{secnumdepth}{2}
\setcounter{tocdepth}{1}

\input{matlabFormating}

% ############  ############
\begin{document}

% ---------- Title ------------
\input{SydeTitle}

% ############ Chapters ############
\pagenumbering{arabic}

\section{Introduction}



\section{Image quality measures}

During the lab, some operations will be applied to a modified image in order to imporve it's appearance. In order to evaluate the quality of the operation, a measure known as the ``Peak Signal to Noise Ratio'' (PSNR) is used, calculated as illustrated in \eqref{eqn-psnr}.

\begin{eqnarray}
\label{eqn-psnr}
PSNR & = & 10 \log_{10}\left ( \frac{{\textup{MAX}_f}^{2}}{\textup{MSE}} \right ) \\
\textup{MSE} & = &\frac{1}{mn} \sum_{i=0}^{m-1}\sum_{j=0}^{n-1} \left \| f(i,j) - g(i,j) \right \|^2
\end{eqnarray}

The Matlab code used to calculate the PSNR for this lab is attached in Appendix~\ref{code-PSNR}.


\section{Digital zooming}

\section{Discrete convolution for image proccessing}

\section{Fourier analyisis}

\section{Point operations}

\appendix
\newpage

\section{Matlab Code}
\subsection{PSNR}
\label{code-PSNR}
\lstinputlisting[language=Matlab]{"matlabFiles/PSNR.m"}


% -------- Bibliography --------
%\addcontentsline{toc}{chapter}{\hspace{13pt} References}
%\bibliography{report}

\end{document}  