\documentclass[article, 1.5space, letterpaper, 12pt, oneside, header, footer]{SydeClass}
\graphicspath{{images/}}
\usepackage{subfigure}
\usepackage{eqnarray}


% --------- Title Info -----------
\titlestyle{design} % used in SydeTitle.tex. Can equal one of the following values: design, work

\title{Lab 1}
\subtitle{Fundamentals of Image Processing}

\coursecode{SYDE 475}
\department{Systems Design Engineering}

\author{Colin Heics, 20240543}
\authorheader{C. Heics}
\authortwo{Neil Sokol, 20265064}
\authorheadertwo{N. Sokol}

\date{\today}
\instructor{Alex Wong}

\subsectionfont{\normalsize}
\setcounter{secnumdepth}{2}
\setcounter{tocdepth}{1}

\usepackage{listings}
\usepackage{color}
\usepackage{textcomp}
\definecolor{listinggray}{gray}{0.9}
\definecolor{lbcolor}{rgb}{0.9,0.9,0.9}
\lstset{
	backgroundcolor=\color{lbcolor},
	tabsize=4,
	rulecolor=,
	language=matlab,
        basicstyle=\scriptsize,
        upquote=true,
        aboveskip={1.5\baselineskip},
        columns=fixed,
        showstringspaces=false,
        extendedchars=true,
        breaklines=true,
        prebreak = \raisebox{0ex}[0ex][0ex]{\ensuremath{\hookleftarrow}},
        frame=single,
        showtabs=false,
        showspaces=false,
        showstringspaces=false,
        identifierstyle=\ttfamily,
        keywordstyle=\color[rgb]{0,0,1},
        commentstyle=\color[rgb]{0.133,0.545,0.133},
        stringstyle=\color[rgb]{0.627,0.126,0.941},
}

% ############  ############
\begin{document}

% ---------- Title ------------

%% Use the command "
%% Use the command "
%% Use the command "\input{SydeTitle}" in your main file to include this file.

\begin{titlepage}
	\makeatletter % use .cls usage for <at>
	
	\pagestyle{empty}
	\equalmargins
	
	\ifthenelse{\equal{\@titlestyle}{work}}{
		\begin{center}
			\vspace*{2em}

			University of Waterloo\\
			Faculty of Engineering\\
			Department of Systems Design Engineering

			\null\vfill
		
			\Huge\@title \\
			\ifdefined \@subtitle \Large\@subtitle \\ \fi
			\normalsize

			\null\vfill
		
			\@company\\
			\@companyaddress \vspace{2em}
		
			\@author\\
			\@date
		\end{center}
	}{\relax} % end if
	
	\ifthenelse{\equal{\@titlestyle}{design}}{
		\begin{center}
			\vspace*{5em}
	
			\Huge\@title \\
			\ifdefined \@subtitle \Large\@subtitle \\ \fi
			\normalsize
	
			\vfill
		
			A Report Submitted in Partial Fulfilment\\
			of the Requirements for \@coursecode \vspace{4em}
		
			\ifdefined \@groupname \@groupname \\ \fi
		  \@author \\
			\ifdefined \@authortwo \@authortwo \\ \fi
			\ifdefined \@authorthree \@authorthree \\ \fi
			\ifdefined \@authorfour \@authorfour \\ \fi
		  \vspace{3em}
		
			Faculty of Engineering \\
			\ifdefined \@department Department of \@department \\ \fi
			\vspace{3em}
		
			\@date \\
			
			\ifdefined \@instructor Course Instructor: \@instructor \\ \fi
			\ifdefined \@supervisor Project Supervisor: \@supervisor \\ \fi
			
		\end{center}
	}{\relax} % end if
	
	\makeatother % return to document usage for <at>
\end{titlepage}

%\pagestyle{plain}
%\offsetmargins" in your main file to include this file.

\begin{titlepage}
	\makeatletter % use .cls usage for <at>
	
	\pagestyle{empty}
	\equalmargins
	
	\ifthenelse{\equal{\@titlestyle}{work}}{
		\begin{center}
			\vspace*{2em}

			University of Waterloo\\
			Faculty of Engineering\\
			Department of Systems Design Engineering

			\null\vfill
		
			\Huge\@title \\
			\ifdefined \@subtitle \Large\@subtitle \\ \fi
			\normalsize

			\null\vfill
		
			\@company\\
			\@companyaddress \vspace{2em}
		
			\@author\\
			\@date
		\end{center}
	}{\relax} % end if
	
	\ifthenelse{\equal{\@titlestyle}{design}}{
		\begin{center}
			\vspace*{5em}
	
			\Huge\@title \\
			\ifdefined \@subtitle \Large\@subtitle \\ \fi
			\normalsize
	
			\vfill
		
			A Report Submitted in Partial Fulfilment\\
			of the Requirements for \@coursecode \vspace{4em}
		
			\ifdefined \@groupname \@groupname \\ \fi
		  \@author \\
			\ifdefined \@authortwo \@authortwo \\ \fi
			\ifdefined \@authorthree \@authorthree \\ \fi
			\ifdefined \@authorfour \@authorfour \\ \fi
		  \vspace{3em}
		
			Faculty of Engineering \\
			\ifdefined \@department Department of \@department \\ \fi
			\vspace{3em}
		
			\@date \\
			
			\ifdefined \@instructor Course Instructor: \@instructor \\ \fi
			\ifdefined \@supervisor Project Supervisor: \@supervisor \\ \fi
			
		\end{center}
	}{\relax} % end if
	
	\makeatother % return to document usage for <at>
\end{titlepage}

%\pagestyle{plain}
%\offsetmargins" in your main file to include this file.

\begin{titlepage}
	\makeatletter % use .cls usage for <at>
	
	\pagestyle{empty}
	\equalmargins
	
	\ifthenelse{\equal{\@titlestyle}{work}}{
		\begin{center}
			\vspace*{2em}

			University of Waterloo\\
			Faculty of Engineering\\
			Department of Systems Design Engineering

			\null\vfill
		
			\Huge\@title \\
			\ifdefined \@subtitle \Large\@subtitle \\ \fi
			\normalsize

			\null\vfill
		
			\@company\\
			\@companyaddress \vspace{2em}
		
			\@author\\
			\@date
		\end{center}
	}{\relax} % end if
	
	\ifthenelse{\equal{\@titlestyle}{design}}{
		\begin{center}
			\vspace*{5em}
	
			\Huge\@title \\
			\ifdefined \@subtitle \Large\@subtitle \\ \fi
			\normalsize
	
			\vfill
		
			A Report Submitted in Partial Fulfilment\\
			of the Requirements for \@coursecode \vspace{4em}
		
			\ifdefined \@groupname \@groupname \\ \fi
		  \@author \\
			\ifdefined \@authortwo \@authortwo \\ \fi
			\ifdefined \@authorthree \@authorthree \\ \fi
			\ifdefined \@authorfour \@authorfour \\ \fi
		  \vspace{3em}
		
			Faculty of Engineering \\
			\ifdefined \@department Department of \@department \\ \fi
			\vspace{3em}
		
			\@date \\
			
			\ifdefined \@instructor Course Instructor: \@instructor \\ \fi
			\ifdefined \@supervisor Project Supervisor: \@supervisor \\ \fi
			
		\end{center}
	}{\relax} % end if
	
	\makeatother % return to document usage for <at>
\end{titlepage}

%\pagestyle{plain}
%\offsetmargins

% ############ Chapters ############
\pagenumbering{arabic}

\section{Introduction}



\section{Image quality measures}

During the lab, some operations will be applied to a modified image in order to imporve it's appearance. In order to evaluate the quality of the operation, a measure known as the ``Peak Signal to Noise Ratio'' (PSNR) is used, calculated as illustrated in \eqref{eqn-psnr}.

\begin{eqnarray}
\label{eqn-psnr}
PSNR & = & 10 \log_{10}\left ( \frac{{\textup{MAX}_f}^{2}}{\textup{MSE}} \right ) \\
\textup{MSE} & = &\frac{1}{mn} \sum_{i=0}^{m-1}\sum_{j=0}^{n-1} \left \| f(i,j) - g(i,j) \right \|^2
\end{eqnarray}

The Matlab code used to calculate the PSNR for this lab is attached in Appendix~\ref{code-PSNR}.


\section{Digital zooming}

\section{Discrete convolution for image proccessing}

\section{Fourier analyisis}

\section{Point operations}

\appendix
\newpage

\section{Matlab Code}
\subsection{PSNR}
\label{code-PSNR}
\lstinputlisting[language=Matlab]{"matlabFiles/PSNR.m"}


% -------- Bibliography --------
%\addcontentsline{toc}{chapter}{\hspace{13pt} References}
%\bibliography{report}

\end{document}  