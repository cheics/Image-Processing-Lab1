\documentclass[article, 1.5space, letterpaper, 12pt, oneside, header, footer]{SydeClass}
\graphicspath{{images/}}
\usepackage{subfigure}
\usepackage{eqnarray}


% --------- Title Info -----------
\titlestyle{design} % used in SydeTitle.tex. Can equal one of the following values: design, work

\title{Lab 1}
\subtitle{Fundamentals of Image Processing}

\coursecode{SYDE 475}
\department{Systems Design Engineering}

\author{Colin Heics, 20240543}
\authorheader{C. Heics}
\authortwo{Neil Sokol, 20265064}
\authorheadertwo{N. Sokol}

\date{\today}
\instructor{Alex Wong}

\subsectionfont{\normalsize}
\setcounter{secnumdepth}{2}
\setcounter{tocdepth}{1}

\input{matlabFormating}

% ############  ############
\begin{document}

% ---------- Title ------------
\input{SydeTitle}

% ############ Chapters ############
\pagenumbering{arabic}

\section{Introduction}



\section{Image quality measures}

During the lab, some operations will be applied to a modified image in order to imporve it's appearance. In order to evaluate the quality of the operation, a measure known as the ``Peak Signal to Noise Ratio'' (PSNR) is used, calculated as illustrated in \eqref{eqn-psnr}.

\begin{eqnarray}
\label{eqn-psnr}
PSNR & = & 10 \log_{10}\left ( \frac{{\textup{MAX}_f}^{2}}{\textup{MSE}} \right ) \\
\textup{MSE} & = &\frac{1}{mn} \sum_{i=0}^{m-1}\sum_{j=0}^{n-1} \left \| f(i,j) - g(i,j) \right \|^2
\end{eqnarray}

The Matlab code used to calculate the PSNR for this lab is attached in Appendix~\ref{code-PSNR}.


\section{Digital zooming}

Digital zooming techniques can introduce artifacts into an image. In order to evaluate several digital zooming techniques, two different sample images will be douwn scaled by a factor of 4, then enlarged using one of three digital zooming techniques. The resulting images will be compared to the ground truth using PSNR, and are included in Figures~\ref{fig:digitalZoom.lena}~and~\ref{fig:digitalZoom.cameraman}.


\begin{figure}[ht]
\centering
	\subfigure[Original image, PSNR $= \infty$]{
	\includegraphics[width=0.45\linewidth]{digitalZoom/lenaBase}
	}
	\subfigure[Nearest Neighbor, PSNR = +33.96 dB]{
	\includegraphics[width=0.45\linewidth]{digitalZoom/lena_NN}
	}
	\subfigure[Bilinear Interpolation, PSNR = +34.26 dB]{
	\includegraphics[width=0.45\linewidth]{digitalZoom/lena_BL}
	}
	\subfigure[Bicubic Interpolation, PSNR = +34.70 dB]{
	\includegraphics[width=0.45\linewidth]{digitalZoom/lena_BC}
	}
	\caption{Various methods of digitally zooming the Lena test image.}
	\label{fig:digitalZoom.lena}
\end{figure}

\begin{figure}[ht]

\centering
	\subfigure[Original image, PSNR $= \infty$]{
	\includegraphics[width=0.45\linewidth]{digitalZoom/cameramanBase}
	}
	\subfigure[Nearest Neighbor, PSNR = +32.91 dB]{
	\includegraphics[width=0.45\linewidth]{digitalZoom/cameraman_NN}
	}
	\subfigure[Bilinear Interpolation, PSNR = +32.62 dB]{
	\includegraphics[width=0.45\linewidth]{digitalZoom/cameraman_BL}
	}
	\subfigure[Bicubic Interpolation, PSNR = +32.88 dB]{
	\includegraphics[width=0.45\linewidth]{digitalZoom/cameraman_BC}
	}
	\caption{Various methods of digitally zooming the Cameraman test image.}
	\label{fig:digitalZoom.cameraman}
\end{figure}



Note that since Matlab already has image resizing functions providing the various digital zooming techniques compared, and the lab instructions have not explicitly asked for a implementation, the Matlab function \texttt{imresize} is used.

\clearpage

\subsection{Discussion questions}

\section{Discrete convolution for image proccessing}

\section{Fourier analyisis}

\section{Point operations}

\appendix
\newpage

\section{Matlab Code}
\subsection{PSNR}
\label{code-PSNR}
\lstinputlisting[language=Matlab]{"matlabFiles/PSNR.m"}


% -------- Bibliography --------
%\addcontentsline{toc}{chapter}{\hspace{13pt} References}
%\bibliography{report}

\end{document}  